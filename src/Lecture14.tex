\begin{lemma} \label{l14_lm7}
	Пусть $b_0,\dots,b_m,\beta_0,\dots,\beta_m \in \mathbb{C}$, пусть $\displaystyle \sum\limits_{k=0}^m b_ke^{\beta_k} = 0$. 
	Рассмотрим многочлен $f(x)=f_n(x) = (x-\beta_0)^n(x-\beta_1)^{n+1}\dots(x-\beta_m)^{n+1}$, пусть $\displaystyle g(x) = \frac{1}{n!}\sum\limits_{l \geq n} f^{(n)}(x)$ (с некоторого $l$ они все станут равны нулю). Тогда
	$$\left| \sum\limits_{k=0}^m b_kg(\beta_k) \right| \leq \frac{c^{n+1}}{n!}, \ \text{ где } c=c(b_0,\dots,b_m,\beta_0,\dots,\beta_m) \text{ -- не зависит от } n.$$
\end{lemma}
\begin{pf}
	Положим $\displaystyle F(x) = \sum\limits_{l \geq 0} f^{(l)}(x)$. Нужно доказать, что $\displaystyle \left| \sum\limits_{k=0}^m b_kF(\beta_k) \right| \leq c^{n+1}$. 
	Заметим, что $\displaystyle F(0)e^{\beta_k} - F(\beta_k) = e^{\beta_k} \int_0^{\beta_k}e^{-z}f(z)dz$ (по частям).\\
	Домножим на $b_k$ и просуммируем по $k$ от $0$ до $m$:\\
	$\displaystyle F(0)\sum_{k=0}^m b_ke^{\beta_k} - \sum\limits_{k=0}^m b_kF(\beta_k) = \sum\limits_{k=0}^m \left[ b_k\int_0^{\beta_k} e^{\beta_k-z}f(z)dz \right]$ -- хотим оценить модуль правой части.
	$$\left| \sum\limits_{k=0}^m \left[ b_k\int_0^{\beta_k} e^{\beta_k-z}f(z)dz \right] \right| \leq \sum\limits_{k=0}^m |b_k|e^r \cdot (2r)^{(m+1)(n+1)} \leq c^{n+1}, \ \text{где } r=\max\limits_{0\leq k \leq m} \lvert \beta_k \rvert$$
	$$\text{для } c=(2r)^{m+1}e^r\cdot\max\left( 1, \, \sum\limits_{k=0}^m \lvert b_k \rvert \right).$$
\end{pf}

\begin{pf} (теоремы об экспоненциальной линейной форме (Т.Э.Л.Ф.)).\\
	Пусть $E$ -- нормальное расширение поля $\mathbb{Q}$, содержащее $a_0,\dots,a_m,\alpha_0,\dots,\alpha_m, \ [E \colon \mathbb{Q}] = \nu$, $\sigma_1,\dots,\sigma_\nu$ -- все автоморфизмы $E$ над $\mathbb{Q}$ (аналогично доказательству теоремы \ref{l13_th10} о Т.Э.Л.Ф. $\Rightarrow$ Т.Л.--В.). 
	Можно считать, что $a_0, \dots, a_m \in \mathbb{Z}_A \ (\in \mathbb{Z}_E)$, так как существует $\tilde{d} \in \mathbb{Z} \backslash \{ 0 \}$ такое, что все $\tilde{d}a_0, \, \tilde{d}a_1, \dots, \tilde{d}a_m \in \mathbb{Z}_A$. От замены то, что дано, и то. что требуется доказать, не поменяется.\\
	Пусть $d \in \mathbb{N}$ -- такое, что $d\alpha_0, \, d\alpha_1, \dots, d\alpha_m \in \mathbb{Z}_E$. Предположим противное: пусть $A(1)=0$. Тогда продлеваем наши автоморфизмы $\sigma_1, \dots, \sigma_\nu$ на $E[[x]]$ как в доказательстве теоремы о Т.Э.Л.Ф $\Rightarrow$ Т.Л.--В. То есть можно рассматривать $(\sigma_iA)(x)$.\\
	Так как $A(x) \in \mathbb{Q}[[x]]$, то $(\sigma_iA)(x) = A(x) \ \forall i = 1,2,\dots,\nu$. Следовательно, $\displaystyle \sum\limits_{k=0}^m \sigma_i (a_k) e^{\sigma_i(\alpha_k)x} = (\sigma_iA)(x) = A(x)$, т.е.
	$$\sum\limits_{k=0}^m \sigma_i(a_k)e^{\sigma_i(\alpha_k)} = A(1) = 0, \quad i = 1,2,\dots,\nu.$$
	Положим $\displaystyle f(x) = f_n(x) = (x-\alpha_0)^n(x-\alpha_1)^{n+1}\dots(x-\alpha_m)^{n+1}, \, g(x) = g_n(x) = \frac{1}{n!} \sum\limits_{l \geq n} f^{(l)}(x)$. 
	Положим $\displaystyle I=I_n = d^{m(n+1)} \sum\limits_{k=0}^m a_kg(\alpha_k)$. Покажем, что $I \in \mathbb{Z}_E$:\\
	$\displaystyle I = \sum\limits_{k=0}^m a_k \sum\limits_{l \geq n} d^{m(n+1)}\frac{1}{n!}f^{(l)}(\alpha_k) = \sum\limits_{k=0}^m a_k \cdot (\text{целое алгебраическое число})$, т.к. $d^{m(n+1)}f(x) = d^{-n}h(dx)$, где $h(t) = (t-d\alpha_0)^n(t-d\alpha_1)^{n+1}\dots(t-d\alpha_m)^{n+1}$ (т.е. $h(t) \in \mathbb{Z}_E$). 
	Следовательно, $\displaystyle d^{m(n+1)}\frac{1}{l!}f^{(l)}(\alpha_k) = d^{-n+l}\frac{1}{l!}h^{(l)}(d\alpha_k) \in \mathbb{Z}_E$ при $l \leq n$. Далее,\\
	$\displaystyle I = d^{m(n+1)}\frac{1}{n!}f^{(n)}(\alpha_0) + (n+1)\sum\limits_{k=0}^m\sum\limits_{l \geq n+1} a_k\frac{l!}{(n+1)!}d^{m(n+1)}\frac{1}{l!}f^{(l)}(\alpha_k) = a_0\prod\limits_{k=1}^m (d\alpha_0-d\alpha_k)^{n+1} + (n+1)J$, где $J \in \mathbb{Z}_E$.\\
	Следовательно, $I \in \mathbb{Z}_E$, причём $I \ne 0$, если $\displaystyle \left( n+1, \, N\left( a_0\prod\limits_{k=1}^m(d\alpha_0 - d\alpha_k) \right) \right) = 1$.\\
	Таких $n$ бесконечно много: $n+1$ -- простое, $\to \infty$. Но тогда и $\sigma_i(I) \in \mathbb{Z}_E$ и $\sigma_i(I) \ne 0$ при "хороших" $n$. Но $\displaystyle \sigma_i(I) = d^{m(n+1)} \sum\limits_{k=0}^m \sigma_i(a_k)g_i(\sigma_i(\alpha_k))$, где $f_i(x) = (\sigma_if)(x) = (x-\sigma_i(\alpha_0))^n(x-\sigma_i(\alpha_1))^{n+1}\dots(x-\sigma_i(\alpha_m))^{n+1}, \, g_i(x) = (\sigma_ig)(x) = \frac{1}{n!} \sum\limits_{l \geq n} f_i^{(l)}(x)$. 
	Применим Лемму \ref{l14_lm7} для $b_k = \sigma_i(a_k), \, \beta_i = \sigma_i(\alpha_k), \, i=1,2,\dots,\nu$. Получим
	$$\left| \sigma_i(I) \right| \leq d^{m(n+1)}\frac{c_i^{n+1}}{n!} \leq \frac{c^{n+1}}{n!}, \, \text{ где } c=d^m\max\limits_i(c_i).$$
	Итак, все $\sigma_i(I) \in \mathbb{Z}_E, \, \sigma_i(I) \ne 0$ при "хороших" $n, \, \sigma_i(I) \to 0$ при $n \to \infty$.\\
	Следовательно, $\displaystyle N(I) = \prod\limits_{i=1}^\nu \sigma_i(I) \to 0$ при $n \to \infty$ и $N(I) \ne 0$ при "хороших" $n$. Но $N(I) \in \mathbb{Z}$! 
	Противоречие.
\end{pf}

\begin{corollary}[из теоремы Л.--В.] \label{l14_cor1}
	Если $\alpha \in \mathbb{A} \backslash \{ 0 \}$, то $e^\alpha \not\in \mathbb{A}$.
\end{corollary}
\begin{pf}
	Пусть $\alpha_0 =0, \, \alpha_1=\alpha$. По теореме Л.--В. $e^{\alpha_0}=1$ и $e^{\alpha_1}=e^\alpha$ линейно независимы над $\mathbb{A}$.
\end{pf}

\begin{corollary} \label{l14_cor2}
	Число $\pi$ -- трансцендентно.
\end{corollary}
\begin{pf}
	Предположим противное. Тогда $\alpha_0 = 0, \, \alpha_1 = i\pi$. По теореме Л.--В. 
	$e^{\alpha_0} = 1, \, e^{\alpha_1} = -1$ линейно независимы над $\mathbb{A}$, но они линейно зависимы. 
	Противоречие.
\end{pf}

\begin{corollary} \label{l14_cor3}
	Если $\alpha \in \mathbb{A} \backslash \{ 1 \}$, то $\ln(\alpha) \not\in \mathbb{A}$.
\end{corollary}

\begin{corollary} \label{l14_cor4}
	Если $\alpha \in \mathbb{A} \backslash \{ 0 \}$, то $\sin(\alpha),\,\cos(\alpha),\,\tg(\alpha) \not\in \mathbb{A}$.
\end{corollary}
\begin{pf}
	$\sin(\alpha) = \frac{1}{2i}e^{i\alpha}-\frac{1}{2i}e^{-i\alpha}$. $i\alpha \ne -i\alpha$ и принадлежит $\mathbb{A} \ \Rightarrow \ $ для $0,\,i\alpha,-i\alpha$ по теореме Л.--В. 
	$1,\,e^{i\alpha},\,e^{-i\alpha}$ ЛНЗ, а если бы $\sin(\alpha) \in \mathbb{A}$, то это было бы ЛЗ.
\end{pf}

\begin{corollary} \label{l14_cor5}
	Если $\beta_1,\dots,\beta_k \in \mathbb{A}$ ЛНЗ над $\mathbb{Q}$, то $e^{\beta_1},\dots,e^{\beta_k}$ -- алгебраически независимы над $\mathbb{A}$.
\end{corollary}
\begin{pf}
	Пусть $f(x_1,\dots,x_k) \in \mathbb{A}[x_1,\dots,x_k]$. Тогда $\displaystyle f(e^{\beta_1},\dots,e^{\beta_k}) = \\ = \sum\limits_{(n_1,\dots,n_k)} a_{n_1\dots n_k}e^{n_1\beta_1+...+n_k\beta_k} =: \alpha_{n_1\dots n_k}$ -- все попарно различны, т.к. 
	$\beta_1,\dots,\beta_k$ ЛНЗ над $\mathbb{Q}$. По теореме Л.--В. 
	$e^{n_1\beta_1+...+n_k\beta_k}$ ЛНЗ над $\mathbb{A}$, следовательно, вся сумма не обращается в ноль.
\end{pf}